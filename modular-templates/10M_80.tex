\documentclass[a4paper,addpoints,12pt,answers]{exam}

%   ============================================================================
%   COMMON STYLE PACKAGES (your .sty files)
%   ============================================================================
\usepackage{sagar-math}
\usepackage{sagar-graphics-tikz}
\usepackage{sagar-boxes}
\usepackage{sagar-exam-core}

%   ============================================================================
%   EXAM-SPECIFIC SETTINGS (change per paper)
%   ============================================================================

% Watermark for this exam (you can change/remove per paper)
\usepackage[firstpageonly]{draftwatermark}
\SetWatermarkText{CAJCS}
\SetWatermarkScale{0.3}
\SetWatermarkLightness{0.95}
\SetWatermarkAngle{45}

%% ============================================================================
%% GRAPHICS PATH (per-machine / per-project)
%% ============================================================================
% Choose the one that matches your current setup:
% \graphicspath{{I:/My Drive/Latex/Images/}} % Windows (Drive)
\graphicspath{{D:/Latex/images/}}            % Windows (local)
% \graphicspath{{G:/Other computers/My Computer (1)/latex/images/}} % Laptop
% \graphicspath{{/media/sagar/0E605F52605F401F/latex/images/}}      % Linux
% \graphicspath{{./Images/}}                                   % Relative

%   ============================================================================
%   DOCUMENT START
%   ============================================================================
\begin{document}

% Header and Footer Details (exam class headers/footers)
\lhead{}
\chead{}
\rhead{}
\lfoot{}
\cfoot{Page \thepage\ of \numpages} % uses \numpages from exam class
\rfoot{}


%\begin{examheaderbox}
\begin{center}
  % \includegraphics[scale=0.4]{clogo} \\
  % \includegraphics[width=0.6\textwidth]{cajcs.JPG}   \\
  \large{\textbf{First Term Examination - September 2022}} \\
\end{center}
\textbf{Sub: Mathematics} \hfill \textbf{Mark: 80} \hspace{1cm} \\
\textbf{Std: 10 ICSE} \hfill \textbf{Time: 3 hrs} \hspace{1cm} \\
\textbf{Date: 05-09-2022} \\
\textbf{\textit{ Reading time : 15 minutes}}

% \vspace{1em}
% \begin{center}
%   \begin{gradetablebox}
%     \cellwidth{0.8cm}
%     \settabletotalpoints{80}
%     \hpword{Marks}
%     \gradetable[h][questions]
%   \end{gradetablebox}
% \end{center}
%\end{examheaderbox}

\hrulefill

% Instructions block
% You can also wrap this in \begin{instructionsbox} ... \end{instructionsbox}
% if you want the styled box from sagar-exam-core.sty
%\begin{instructionsbox}
\begin{center}
  \begin{itemize}[leftmargin=0pt, itemsep=0.3em]
    \item Answers to this must be written on the paper provided separately.
    \item You will not be allowed to write during first 15 minutes.
    \item This time is to be spent in reading the question paper.
    \item The time given at the head of this paper is the time allowed for writing answers.
    \item This question paper consists of \textbf{\numpages} printed pages.
    \item[\textbf{•}] Attempt \textbf{all} questions from \textbf{Section A} and \textbf{any four} questions from \textbf{Section B}.
    \item All working, including rough work, must be clearly shown, and must be done on the same sheet as the rest of the answer.
    \item Omission of essential working will result in loss of marks.
    \item The intended marks for questions or parts of questions are given in brackets [ ].
    \item Mathematical tables are provided.
  \end{itemize}

  \vspace{1em}
  \textbf{Instructions for Supervising Examiner:}\\
  Kindly read aloud the instructions given above to all candidates present in the examination hall.
\end{center}
%\end{instructionsbox}

\hrulefill

% You can also use:
% \sectionheader{A}{Attempt all questions}
\begin{center}
  \textbf{SECTION A} \\
  \textit{Attempt all questions}
\end{center}

%======================== Questions Begin ============================
\begin{questions}
  \doublespacing
  \pointsdroppedatright
  \marksnotpoints
  %\marginpointname{ \points}
  \setlength{\rightpointsmargin}{2.1cm}
  \pointformat{\bfseries\boldmath[\themarginpoints]}

  % Question 1
  \question[15] Choose the correct answers to the questions from the given options. \droppoints
  \begin{parts}
    \renewcommand{\thepartno}{\roman{partno}}
    %MCQ 1
    \part this is new file.

    \begin{oneparchoices}
      \choice
      \choice
      \choice
      \choice
    \end{oneparchoices}
    %MCQ 2
    \part

    \begin{oneparchoices}
      \choice
      \choice
      \choice
      \choice
    \end{oneparchoices}
    %MCQ 3
    \part

    \begin{oneparchoices}
      \choice
      \choice
      \choice
      \choice
    \end{oneparchoices}
    %MCQ 4
    \part

    \begin{oneparchoices}
      \choice
      \choice
      \choice
      \choice
    \end{oneparchoices}
    %MCQ 5
    \part

    \begin{oneparchoices}
      \choice
      \choice
      \choice
      \choice
    \end{oneparchoices}
    %MCQ 6
    \part

    \begin{oneparchoices}
      \choice
      \choice
      \choice
      \choice
    \end{oneparchoices}
    %MCQ 7
    \part

    \begin{oneparchoices}
      \choice
      \choice
      \choice
      \choice
    \end{oneparchoices}
    %MCQ 8
    \part

    \begin{oneparchoices}
      \choice
      \choice
      \choice
      \choice
    \end{oneparchoices}
    %MCQ 9
    \part

    \begin{oneparchoices}
      \choice
      \choice
      \choice
      \choice
    \end{oneparchoices}
    %MCQ 10
    \part

    \begin{oneparchoices}
      \choice
      \choice
      \choice
      \choice
    \end{oneparchoices}
    % 11
    \part

    \begin{oneparchoices}
      \choice
      \choice
      \choice
      \choice
    \end{oneparchoices}
    % 12
    \part

    \begin{oneparchoices}
      \choice
      \choice
      \choice
      \choice
    \end{oneparchoices}
    % 13
    \part

    \begin{oneparchoices}
      \choice
      \choice
      \choice
      \choice
    \end{oneparchoices}
    % 14
    \part

    \begin{oneparchoices}
      \choice
      \choice
      \choice
      \choice
    \end{oneparchoices}
    % 15
    \part

    \begin{oneparchoices}
      \choice
      \choice
      \choice
      \choice
    \end{oneparchoices}
  \end{parts}

  % Question 2
  \question
  \begin{parts}
    \part[4]    \droppoints
    \part[4]    \droppoints
    \part[4]    \droppoints
  \end{parts}

  % Question 3
  \question
  \begin{parts}
    \part[4]    \droppoints
    \part[4]    \droppoints
    \part[4]    \droppoints
  \end{parts}

  % Section B Header
  % \sectionheader{B}{Attempt any 4 questions (40 marks)}
  \begin{center}
    \textbf{SECTION B (40 marks)} \\
    \textbf{Attempt any 4 questions }
  \end{center}

  % Question 4
  \question
  \begin{parts}
    \part[3]    \droppoints
    \part[3]    \droppoints
    \part[4]    \droppoints
  \end{parts}

  % Question 5
  \question
  \begin{parts}
    \part[3]    \droppoints
    \part[3]    \droppoints
    \part[4]    \droppoints
  \end{parts}

  % Question 6
  \question
  \begin{parts}
    \part[3]    \droppoints
    \part[3]    \droppoints
    \part[4]    \droppoints
  \end{parts}

  % Question 7
  \question
  \begin{parts}
    \part[3]    \droppoints
    \part[3]    \droppoints
    \part[4]    \droppoints
  \end{parts}

  % Question 8
  \question
  \begin{parts}
    \part[3]    \droppoints
    \part[3]    \droppoints
    \part[4]    \droppoints
  \end{parts}

  % Question 9
  \question
  \begin{parts}
    \part[3]    \droppoints
    \part[3]    \droppoints
    \part[4]    \droppoints
  \end{parts}

  % Question 10
  \question
  \begin{parts}
    \part[4]    \droppoints
    \part[6]    \droppoints
  \end{parts}

\end{questions}

\centering
\vspace{2cm}
********

\end{document}
