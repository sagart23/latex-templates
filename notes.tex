\documentclass[a4paper,12pt,twoside,openright]{article}

%   ============================================================================
%   COMMON STYLE PACKAGES (correct loading order)
%   ============================================================================
\usepackage{sagar-math}                    % Math, physics, siunitx, rupee
\usepackage{sagar-graphics-tikz-core}      % TikZ, pgfplots, graphics
\usepackage{sagar-boxes}                   % Color definitions, tcolorboxes
\usepackage{sagar-article-core}            % Article geometry & formatting

%   ============================================================================
%   LIST CUSTOMIZATION (optional)
%   ============================================================================
\usepackage{enumitem}
\setlist[enumerate]{label=(\alph*)}
\setlist[itemize]{label=\textbullet}

%   ============================================================================
%   WATERMARK (after geometry is set)
%   ============================================================================
\usepackage[firstpageonly]{draftwatermark}
\SetWatermarkText{CAJCS Senior Section}
\SetWatermarkScale{0.45}
\SetWatermarkLightness{0.92}
\SetWatermarkAngle{45}

%   ============================================================================
%   GRAPHICS PATH
%   ============================================================================
\graphicspath{{./Images/}}

%   ============================================================================
%   HEADERS & FOOTERS
%   ============================================================================
\usepackage{fancyhdr}
\usepackage{lastpage}

\pagestyle{fancy}
\fancyhf{}

% Header
\fancyhead[LE]{\sffamily\bfseries\color{primaryblue}\leftmark}
\fancyhead[RO]{\sffamily\bfseries\color{primaryblue}\rightmark}
\fancyhead[LO,RE]{\sffamily\small The Cathedral \& John Connon School}

% Footer
\fancyfoot[LE,RO]{\sffamily\bfseries Page \thepage\ of \pageref{LastPage}}
\fancyfoot[LO,RE]{\sffamily\small Physics \& Mathematics Department}

% Thin blue rule under header
\renewcommand{\headrulewidth}{0.5pt}
\renewcommand{\headrule}{%
  \hbox to\headwidth{%
    \color{primaryblue}\leaders\hrule height \headrulewidth\hfill}}

%   ============================================================================
%   SECTION STYLING
%   ============================================================================
\usepackage{sectsty}
\sectionfont{\sffamily\Large\bfseries\color{primaryblue}}
\subsectionfont{\sffamily\large\bfseries\color{secondarygreen!80!black}}
\subsubsectionfont{\sffamily\normalsize\bfseries\color{gray!70!black}}

%   ============================================================================
%   DOCUMENT METADATA
%   ============================================================================
\title{Physics \& Mathematics Notes}
\author{Sagar Tamhankar}
\date{\today}

%   ============================================================================
%   HYPERREF (always last)
%   ============================================================================
\usepackage{hyperref}
\hypersetup{
    pdfauthor   = {Sagar Tamhankar},
    pdftitle    = {Physics \& Mathematics Educational Notes},
    colorlinks  = true,
    linkcolor   = primaryblue,
    citecolor   = secondarygreen,
    urlcolor    = primaryblue,
    filecolor   = accentorange,
    pdfstartview = FitH
}

%   ============================================================================
%   CONDITIONAL COMPILATION FLAGS
%   ============================================================================
\newif\ifteacher
\teacherfalse   % Change to \teachertrue for teacher version

\newif\ifhint
\hintfalse      % Change to \hinttrue to show hints

\newif\ifanswer
\answerfalse    % Change to \answertrue to show answers

%   ============================================================================
%   DOCUMENT BEGIN
%   ============================================================================
\begin{document}

% Uncomment if you want a title page
% \maketitle
% \thispagestyle{empty}
% \newpage

%   ============================================================================
%   CONTENT
%   ============================================================================

\section{Introduction to Logarithms}

\begin{definition}{Logarithm}
A logarithm answers the question:
\emph{"To what power must the base be raised to obtain a given number?"}

Formally, if $b^x = y$, then $\log_b y = x$.
\end{definition}

\begin{noteBox}
The base of the logarithm determines the exponential scale. Common bases include:
\begin{itemize}
  \item Base 10 (common logarithm): $\log_{10} x$ or simply $\log x$
  \item Base $e$ (natural logarithm): $\log_e x$ or $\ln x$
  \item Base 2 (binary logarithm): $\log_2 x$
\end{itemize}
\end{noteBox}

\subsection{Laws of Logarithms}

\begin{theorem}{Product Law}
For positive numbers $a$ and $b$:
\[ \log_c(ab) = \log_c a + \log_c b \]
\end{theorem}

\begin{theorem}{Quotient Law}
For positive numbers $a$ and $b$:
\[ \log_c\left(\frac{a}{b}\right) = \log_c a - \log_c b \]
\end{theorem}

\begin{theorem}{Power Law}
For positive number $a$ and real number $n$:
\[ \log_c(a^n) = n \cdot \log_c a \]
\end{theorem}

\subsection{Worked Examples}

\begin{example}{Evaluating Logarithms}
Find the value of $\log_2 16$.

\begin{Mysolutionbox}
We need to find $x$ such that $2^x = 16$.

Since $2^4 = 16$, we have $\log_2 16 = 4$.
\end{Mysolutionbox}
\end{example}

\begin{example}{Using Logarithm Laws}
Simplify: $\log_3 27 + \log_3 9 - \log_3 81$

\begin{Mysolutionbox}
\begin{align*}
\log_3 27 + \log_3 9 - \log_3 81 &= \log_3(3^3) + \log_3(3^2) - \log_3(3^4) \\
&= 3 + 2 - 4 \\
&= 1
\end{align*}

Alternatively, using logarithm laws:
\begin{align*}
\log_3 27 + \log_3 9 - \log_3 81 &= \log_3\left(\frac{27 \times 9}{81}\right) \\
&= \log_3 3 \\
&= 1
\end{align*}
\end{Mysolutionbox}
\end{example}

\begin{important}
Always check that the argument of a logarithm is positive. The expression $\log_b x$ is only defined for $x > 0$ and $b > 0$, $b \neq 1$.
\end{important}

\begin{errorbox}
\textbf{Common Mistake:} Students often write $\log(a + b) = \log a + \log b$.

This is \emphred{INCORRECT}! The correct law is:
\[ \log(ab) = \log a + \log b \quad \text{(Product Law)} \]

There is no simplification for $\log(a + b)$.
\end{errorbox}

\section{Applications}

Logarithms appear in many real-world contexts:
\begin{itemize}
  \item \emphblue{pH Scale} in chemistry: $\text{pH} = -\log_{10}[\text{H}^+]$
  \item \emphblue{Richter Scale} for earthquakes
  \item \emphblue{Decibel Scale} for sound intensity
  \item \emphblue{Compound Interest} calculations
\end{itemize}

\begin{teachingtip}
When teaching logarithms, always connect back to exponentials. Students understand $2^3 = 8$ intuitively, so build on that to explain $\log_2 8 = 3$ as the inverse operation.
\end{teachingtip}

\end{document}
