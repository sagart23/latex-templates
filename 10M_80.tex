\documentclass[a4paper,addpoints,12pt,answers]{exam}

%   ============================================================================
%   COMMON STYLE PACKAGES (correct loading order)
%   ============================================================================
\usepackage{sagar-math}                    % Math, physics, siunitx, rupee
\usepackage{sagar-graphics-tikz-core}      % TikZ, pgfplots, graphics
\usepackage{sagar-boxes}                   % Color definitions, tcolorboxes
\usepackage{sagar-exam-core}               % Exam geometry & formatting

%   ============================================================================
%   WATERMARK (after geometry is set)
%   ============================================================================
\usepackage[firstpageonly]{draftwatermark}
\SetWatermarkText{CAJCS}
\SetWatermarkScale{0.3}
\SetWatermarkLightness{0.95}
\SetWatermarkAngle{45}

%   ============================================================================
%   GRAPHICS PATH
%   ============================================================================
\graphicspath{{./Images/}}

%   ============================================================================
%   HYPERREF (always last)
%   ============================================================================
\usepackage{hyperref}
\hypersetup{
    pdfauthor   = {Sagar Tamhankar},
    pdftitle    = {ICSE Mathematics Exam},
    pdfsubject  = {Mathematics - Standard 10},
    colorlinks  = true,
    linkcolor   = primaryblue,
    urlcolor    = primaryblue,
    citecolor   = secondarygreen
}

%   ============================================================================
%   DOCUMENT START
%   ============================================================================
\begin{document}

% Header and Footer (exam class commands)
\lhead{}
\chead{}
\rhead{}
\lfoot{}
\cfoot{Page \thepage\ of \numpages}
\rfoot{}

% School Header
\begin{center}
   \includegraphics[scale=0.4]{clogo} \\
   \includegraphics[width=0.6\textwidth]{cajcs.JPG} \\
   \large{\textbf{First Term Examination - September 2022}} \\
\end{center}

% Exam Details
\noindent
\textbf{Sub: Mathematics} \hfill \textbf{Mark: 80} \\
\textbf{Std: 10 ICSE} \hfill \textbf{Time: 3 hrs} \\
\textbf{Date: 05-09-2022}

\hrulefill

% Instructions
\begin{center}
  \begin{itemize}[leftmargin=0pt, itemsep=0.3em]
    \item Answers to this must be written on the paper provided separately.
    \item You will not be allowed to write during first 15 minutes.
    \item This time is to be spent in reading the question paper.
    \item The time given at the head of this paper is the time allowed for writing answers.
    \item This question paper consists of \textbf{\numpages} printed pages.
    \item[\textbf{•}] Attempt \textbf{all} questions from \textbf{Section A} and \textbf{any four} questions from \textbf{Section B}.
    \item All working, including rough work, must be clearly shown, and must be done on the same sheet as the rest of the answer.
    \item Omission of essential working will result in loss of marks.
    \item The intended marks for questions or parts of questions are given in brackets [ ].
    \item Mathematical tables are provided.
  \end{itemize}

  \vspace{1em}
  \textbf{Instructions for Supervising Examiner:}\\
  Kindly read aloud the instructions given above to all candidates present in the examination hall.
\end{center}

\hrulefill

%   ============================================================================
%   SECTION A
%   ============================================================================
\begin{center}
  \textbf{SECTION A} \\
  \textit{Attempt all questions}
\end{center}

\begin{questions}
  \doublespacing
  \pointsdroppedatright
  \marksnotpoints
  \setlength{\rightpointsmargin}{2.1cm}
  \pointformat{\bfseries\boldmath[\themarginpoints]}

  % Question 1 - Multiple Choice (15 marks)
  \question[15] Choose the correct answers to the questions from the given options. \droppoints
  \begin{parts}
    \renewcommand{\thepartno}{\roman{partno}}
    
    % MCQ 1
    \part This is new file.
    \begin{oneparchoices}
      \choice Option A
      \choice Option B
      \choice Option C
      \choice Option D
    \end{oneparchoices}
    
    % MCQ 2
    \part 
    \begin{oneparchoices}
      \choice Option A
      \choice Option B
      \choice Option C
      \choice Option D
    \end{oneparchoices}
    
    % MCQ 3
    \part 
    \begin{oneparchoices}
      \choice Option A
      \choice Option B
      \choice Option C
      \choice Option D
    \end{oneparchoices}
    
    % MCQ 4
    \part 
    \begin{oneparchoices}
      \choice Option A
      \choice Option B
      \choice Option C
      \choice Option D
    \end{oneparchoices}
    
    % MCQ 5
    \part 
    \begin{oneparchoices}
      \choice Option A
      \choice Option B
      \choice Option C
      \choice Option D
    \end{oneparchoices}
    
    % MCQ 6
    \part 
    \begin{oneparchoices}
      \choice Option A
      \choice Option B
      \choice Option C
      \choice Option D
    \end{oneparchoices}
    
    % MCQ 7
    \part 
    \begin{oneparchoices}
      \choice Option A
      \choice Option B
      \choice Option C
      \choice Option D
    \end{oneparchoices}
    
    % MCQ 8
    \part 
    \begin{oneparchoices}
      \choice Option A
      \choice Option B
      \choice Option C
      \choice Option D
    \end{oneparchoices}
    
    % MCQ 9
    \part 
    \begin{oneparchoices}
      \choice Option A
      \choice Option B
      \choice Option C
      \choice Option D
    \end{oneparchoices}
    
    % MCQ 10
    \part 
    \begin{oneparchoices}
      \choice Option A
      \choice Option B
      \choice Option C
      \choice Option D
    \end{oneparchoices}
    
    % MCQ 11
    \part 
    \begin{oneparchoices}
      \choice Option A
      \choice Option B
      \choice Option C
      \choice Option D
    \end{oneparchoices}
    
    % MCQ 12
    \part 
    \begin{oneparchoices}
      \choice Option A
      \choice Option B
      \choice Option C
      \choice Option D
    \end{oneparchoices}
    
    % MCQ 13
    \part 
    \begin{oneparchoices}
      \choice Option A
      \choice Option B
      \choice Option C
      \choice Option D
    \end{oneparchoices}
    
    % MCQ 14
    \part 
    \begin{oneparchoices}
      \choice Option A
      \choice Option B
      \choice Option C
      \choice Option D
    \end{oneparchoices}
    
    % MCQ 15
    \part 
    \begin{oneparchoices}
      \choice Option A
      \choice Option B
      \choice Option C
      \choice Option D
    \end{oneparchoices}
  \end{parts}

  % Question 2 (12 marks total)
  \question
  \begin{parts}
    \part[4] \droppoints
    \part[4] \droppoints
    \part[4] \droppoints
  \end{parts}

  % Question 3 (12 marks total)
  \question
  \begin{parts}
    \part[4] \droppoints
    \part[4] \droppoints
    \part[4] \droppoints
  \end{parts}

  %   ==========================================================================
  %   SECTION B (40 marks)
  %   ==========================================================================
  \begin{center}
    \textbf{SECTION B (40 marks)} \\
    \textbf{Attempt any 4 questions}
  \end{center}

  % Question 4 (10 marks)
  \question
  \begin{parts}
    \part[3] \droppoints
    \part[3] \droppoints
    \part[4] \droppoints
  \end{parts}

  % Question 5 (10 marks)
  \question
  \begin{parts}
    \part[3] \droppoints
    \part[3] \droppoints
    \part[4] \droppoints
  \end{parts}

  % Question 6 (10 marks)
  \question
  \begin{parts}
    \part[3] \droppoints
    \part[3] \droppoints
    \part[4] \droppoints
  \end{parts}

  % Question 7 (10 marks)
  \question
  \begin{parts}
    \part[3] \droppoints
    \part[3] \droppoints
    \part[4] \droppoints
  \end{parts}

  % Question 8 (10 marks)
  \question
  \begin{parts}
    \part[3] \droppoints
    \part[3] \droppoints
    \part[4] \droppoints
  \end{parts}

  % Question 9 (10 marks)
  \question
  \begin{parts}
    \part[3] \droppoints
    \part[3] \droppoints
    \part[4] \droppoints
  \end{parts}

  % Question 10 (10 marks)
  \question
  \begin{parts}
    \part[4] \droppoints
    \part[6] \droppoints
  \end{parts}

\end{questions}

\centering
\vspace{2cm}
********

\end{document}
