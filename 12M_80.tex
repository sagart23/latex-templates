% !TeX document class: article
% !TeX spellcheck: en_US
%
% File: 11M_Functions_Notes_20250819.tex
% Description: Notes on the properties of various types of functions, including polynomial and exponential.
% Keywords: math, functions, algebra, polynomial, exponential
% Date: August 19, 2025
% Author: [Your Name]
\documentclass[a4paper, addpoints, 12pt]{exam}

% ----- Encoding and Font Setup (for pdfLaTeX) -----
% Use T1 font encoding for better hyphenation and support for various characters.
\usepackage[T1]{fontenc}
% Input encoding for your source files (most modern editors use UTF-8).
\usepackage[utf8]{inputenc}
% Latin Modern is a polished version of Computer Modern and is good for pdfLaTeX.
\usepackage{lmodern}

% ----- Microtype for Enhanced Typography -----
% microtype provides sophisticated font expansion, protrusion, and tracking.
% [final] ensures it's always active, even in draft mode.
\usepackage[final]{microtype}
\microtypesetup{protrusion=true, expansion=true, tracking=true}
% Apply microtype settings specifically to math for better spacing.
\UseMicrotypeSet[protrusion]{basicmath}

% ----- Page Geometry -----
% Using slightly more generous margins for better readability and print safety
% while still being space-efficient for question papers.
% 2cm all around is a good balance for question papers.
\usepackage[left=2cm, right=2cm, top=2.54cm, bottom=2.54cm]{geometry}

%================================
% Document Metadata (Consolidated)
% Using \def instead of \newcommand for redefinability if needed
%================================
\def\docauthor{Sagar Tamhankar}
\def\doctitle{12 ISC Math 80 Question Paper}
\def\docsubject{Mathematics}
\def\dockeywords{LaTeX, Mathematics, ISC}

\title{\doctitle}
\author{\docauthor}
\date{\today}

%================================
% Core Packages (Consolidated & Cleaned)
%================================
% amsmath is essential for advanced math typesetting. fleqn for left-aligned equations.
\usepackage[fleqn]{amsmath}
% amsfonts (for Blackboard Bold, etc.) and amssymb (for more math symbols).
\usepackage{amsfonts, amssymb}
% Comprehensive package for including graphics.
\usepackage{graphicx}
% For multi-column layouts (useful for worksheets/quizzes).
\usepackage{multicol}
% For wrapping text around figures.
\usepackage{wrapfig}
% For flexible table columns, extended tables, and variable width content.
\usepackage{tabularx, longtable, varwidth}
% For professional-looking horizontal rules in tables.
\usepackage{booktabs}
% Improved floating environments for figures/tables.
\usepackage{float}
% For advanced captioning and sub-captions.
\usepackage{caption, subcaption}
% For customizable list environments (enumerate, itemize).
\usepackage{enumitem}
% For adding temporary notes/reminders.
\usepackage{todonotes}
% Comprehensive physics macros and environments (e.g., \qty, \vec, \abs).
\usepackage{physics}
% For generating QR codes.
\usepackage{qrcode}
% For commenting out blocks of text/code.
\usepackage{comment}
% For adjusting line spacing (used later with \doublespacing or \setstretch).
\usepackage{setspace}
% For colors, using xcolor for advanced color models.
\usepackage{xcolor} % xcolor is generally preferred over color
% For the Indian Rupee symbol.
\usepackage{tfrupee} % Placed with other general utility packages

%================================
% Fancy Environments: Boxes & Colors (tcolorbox)
%================================
% Comprehensive box package. 'most' loads many libraries, 'breakable' allows boxes to span pages.
\usepackage[most, many, breakable]{tcolorbox}
\tcbuselibrary{skins}

%================================
% TikZ & PGFPlots
%================================
\usepackage{tikz}
\usepackage{tkz-euclide}
\usepackage{pgfplots, pgfplotstable}
\usepackage[americanvoltages,fulldiodes,siunitx]{circuitikz}
\usetikzlibrary{patterns, arrows.meta, calc, arrows, shadows.blur, math, angles, shapes.geometric}
% Setting compat to 1.18 for consistency with other preambles.
\pgfplotsset{compat=1.18}
\usepgfplotslibrary{statistics, fillbetween}
\tikzset{>=latex} % Set default arrow style for TikZ

%================================
% Custom Math Operators
%================================
\DeclareMathOperator\cosec{cosec}
%================================
% Exam Class Specific Customizations
%================================
% Styling for question, part, and subpart labels.
\renewcommand{\questionlabel}{\textbf{Question \thequestion.}}
\renewcommand{\partlabel}{\textbf{(\thepartno)}}
\renewcommand{\subpartlabel}{\textbf{\thesubpart.}}

%================================
% Watermark
%================================
\usepackage[firstpageonly]{draftwatermark}
\SetWatermarkText{CAJCS Senior Section}
\SetWatermarkScale{0.5}

%================================
% PDF Metadata & Hyperref Settings
%================================
\usepackage{hyperref} % Moved up with other core packages for consistency
\hypersetup{
    pdfauthor={\docauthor},
    pdftitle={\doctitle},
    pdfsubject={\docsubject},
    pdfkeywords={\dockeywords},
    colorlinks=true,
    linkcolor=blue,    % Keep links blue for question papers
    urlcolor=blue,
    citecolor=blue,
    bookmarks=true,         % Enable bookmarks
    bookmarksopen=true,     % Bookmarks panel open by default
    bookmarksopenlevel=2,   % Open bookmarks to level 2 (sections)
    pdfstartview=FitH,      % Fit page to window width on opening
    % hidelinks, % Uncomment this if you want links but no visible boxes/colors
    linktoc=all             % Make link to entire section clickable (not just number)
}

%================================
% Graphics Path
%================================
% Consolidate to one active path.
\graphicspath{{./Images/}} % Prefer relative path first, then absolute if needed.
% \graphicspath{{I:/My Drive/Latex/Images/}} % Windows
% \graphicspath{{D:/Latex/images/}} % Windows
% \graphicspath{{G:/Other computers/My Computer (1)/latex/images/}} % Laptop
% \graphicspath{{/media/sagar/0E605F52605F401F/latex/images/}} % Linux

% ----------- Readability & Layout Enhancements ------------
% Consistent readability settings for the main text flow.
% \doublespacing is used within the questions environment, so this affects surrounding text.
\linespread{1.15}            % Improves overall line spacing for general text
\setlength{\parskip}{0.5em}    % Adds space between paragraphs
\setlength{\parindent}{0pt}    % No paragraph indentation
\flushbottom                  % Equalizes page height on two-page spreads

%=====================Start of document ===========================
\begin{document}
% Header and Footer Details
\lhead{}
\chead{}
\rhead{}
\lfoot{}
\cfoot{Page \thepage\ of \numpages}
\rfoot{}

\noindent
\begin{center}
%	\includegraphics[scale=0.4]{clogo} \\\
%	\includegraphics[width=0.6\textwidth]{cajcs.JPG}   \\\
	\large{\textbf{First Term Examination - September 2022}}  \\\
\end{center}
\textbf{Sub: Mathematics} \hfill \textbf{Mark: 80} \hspace{1cm} \\\
\textbf{Std: 12 ISC} \hfill \textbf{Time: 3 hrs} \hspace{1cm} \\\
\textbf{Date: 05-09-2022} \\\
\textbf{\textit{ Reading time : 15 minutes}}

\hrulefill
\begin{center}
	You are allowed an additional 15 minutes for only reading the paper. \\\
	You must \textbf{NOT} start writing during reading time. \\\
	The question paper has \pageref{lastpage} printed pages. \\\
	The Question Paper is divided into \textbf{three} sections and has \textbf{22 questions} in all. \\\
	\textbf{Section A} is compulsory and has \textbf{fourteen} questions.
	You are required to attempt \textbf{all} questions either from \textbf{Section B} or \textbf{Section C}. \\\
	\textbf{Section B} and \textbf{Section C} have \textbf{four} questions each. \\\
	Internal choices have been provided in two questions of 2 marks, two
	questions of 4 marks and two questions of 6 marks in Section A. \\\
	Internal choices have been provided in one question of 2 marks and one
	question of 4 marks each in Section B and Section C.\\\
	While attempting \textbf{Multiple Choice Questions in Section A, B and C}, you
	are required to \textbf{write only ONE option as the answer}. \\\
	The intended marks for questions or parts of questions are given in the brackets []. \\\
	All workings, including rough work, should be done on the same page as, and
	adjacent to, the rest of the answer. \\\
	Mathematical tables and graph papers are provided.
\end{center}
\hrulefill
\begin{center}
	\textbf{SECTION A} \\\
	\textit{Attempt all questions}
\end{center}
%======================== Questions Begin ============================
\begin{questions}
	\doublespacing
	\pointsdroppedatright
	\marksnotpoints
	%\marginpointname{ \points}
	\setlength{\rightpointsmargin}{2.1cm}
	\pointformat{\bfseries\boldmath[\themarginpoints]}
	% Question 1
	\question[15] In sub parts (i) to (x) choose the correct option and in sub-part (xi) to (xv) answer the questions as instructed. \droppoints
	\begin{parts}
		\renewcommand{\thepartno}{\roman{partno}}
		%MCQ 1
		\part

		\begin{oneparchoices}
			\choice
			\choice
			\choice
			\choice
		\end{oneparchoices}
		%MCQ 2		
		\part

		\begin{oneparchoices}
			\choice
			\choice
			\choice
			\choice
		\end{oneparchoices}
		%MCQ 3		
		\part

		\begin{oneparchoices}
			\choice
			\choice
			\choice
			\choice
		\end{oneparchoices}
		%MCQ 4		
		\part

		\begin{oneparchoices}
			\choice
			\choice
			\choice
			\choice
		\end{oneparchoices}
		%MCQ 5	
		\part

		\begin{oneparchoices}
			\choice
			\choice
			\choice
			\choice
		\end{oneparchoices}
		%MCQ 6
		\part

		\begin{oneparchoices}
			\choice
			\choice
			\choice
			\choice
		\end{oneparchoices}
		%MCQ 7		
		\part

		\begin{oneparchoices}
			\choice
			\choice
			\choice
			\choice
		\end{oneparchoices}
		%MCQ 8		
		\part

		\begin{oneparchoices}
			\choice
			\choice
			\choice
			\choice
		\end{oneparchoices}
		%MCQ 9		
		\part

		\begin{oneparchoices}
			\choice
			\choice
			\choice
			\choice
		\end{oneparchoices}
		%MCQ 10	
		\part

		\begin{oneparchoices}
			\choice
			\choice
			\choice
			\choice
		\end{oneparchoices}
		% 11
		\part
		% 12
		\part
		% 13
		\part
		% 14
		\part
		% 15
		\part
	\end{parts}

	% Question 2
	\question[2]
	\begin{parts}
		\part    \droppoints

		\begin{center}
			\textbf{OR}
		\end{center}
		\part    \droppoints
	\end{parts}
	% Question 3
	\question[2]     \droppoints
	% Question 4
	\question[2]     \droppoints

	% Question 5
	\question[2]
	\begin{parts}
		\part    \droppoints

		\begin{center}
			\textbf{OR}
		\end{center}
		\part    \droppoints
	\end{parts}
	% Question 6
	\question[2]     \droppoints
	% Question 7
	\question[4]    \droppoints
	% Question 8
	\question[4]    \droppoints
	% Question 9
	\question[4]
	\begin{parts}
		\part     \droppoints

		\begin{center}
			\textbf{OR}
		\end{center}
		\part    \droppoints
	\end{parts}
	% Question 10
	\question[4]
	\begin{parts}
		\part    \droppoints
		\begin{center}
			\textbf{OR}
		\end{center}
		\part    \droppoints
	\end{parts}
	% Question 11
	\question[6]    \droppoints
	% Question 12
	\question[6]
	\begin{parts}
		\part    \droppoints
		\begin{center}
			\textbf{OR}
		\end{center}
		\part    \droppoints
	\end{parts}
	% Question 13
	\question[6]
	\begin{parts}
		\part    \droppoints

		\begin{center}
			\textbf{OR}
		\end{center}
		\part    \droppoints
	\end{parts}
	% Question 14
	\question[6]    \droppoints

	\begin{center}
		\textbf{SECTION B - 15 marks}
	\end{center}
	% Question 15		
	\question[5] In sub-parts (a) and (b) choose the correct options and in sub-parts (c) and (d) answer the questions as instructed. \droppoints
	\begin{parts}
		\part

		\begin{oneparchoices}
			\choice
			\choice
			\choice
			\choice
		\end{oneparchoices}

		\part

		\begin{oneparchoices}
			\choice
			\choice
			\choice
			\choice
		\end{oneparchoices}
		\part

		\begin{oneparchoices}
			\choice
			\choice
			\choice
			\choice
		\end{oneparchoices}

		\part
		\part
	\end{parts}
	% Question 16
	\question
	\begin{parts}
		\part[2]    \droppoints

		\begin{center}
			\textbf{OR}
		\end{center}
		\part[2]     \droppoints
	\end{parts}

	% Question 17
	\question
	\begin{parts}
		\part[4]   \droppoints
		\begin{center}
			\textbf{OR}
		\end{center}
		\part[4]   		   \droppoints
	\end{parts}
	% Question 18
	\question[4]    \droppoints

	\begin{center}
		\textbf{SECTION C - 15 marks}
	\end{center}
	% Question 19		
	\question[5] In sub-parts (a) and (b) choose the correct options and in sub-parts (c) and (d) answer the questions as instructed. \droppoints
	\begin{parts}
		\part

		\begin{oneparchoices}
			\choice
			\choice
			\choice
			\choice
		\end{oneparchoices}

		\part

		\begin{oneparchoices}
			\choice
			\choice
			\choice
			\choice
		\end{oneparchoices}
		\part

		\begin{oneparchoices}
			\choice
			\choice
			\choice
			\choice
		\end{oneparchoices}
		\part
		\part
	\end{parts}
	% Question 20
	\question
	\begin{parts}
		\part[2]    \droppoints

		\begin{center}
			\textbf{OR}
		\end{center}
		\part[2]     \droppoints
	\end{parts}

	% Question 21
	\question
	\begin{parts}
		\part[4]   \droppoints

		\begin{center}
			\textbf{OR}
		\end{center}
		\part[4]   		   \droppoints
	\end{parts}
	% Question 22
	\question[4]    \droppoints

\end{questions}
\centering
\vspace{2cm}
********
\label{lastpage}
\end{document}
