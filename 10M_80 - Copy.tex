%   ============================================================================
%   TEMPLATE 3: ENHANCED EXAM QUESTION PAPER
%   ============================================================================
%  !TeX document class: exam
%  !TeX spellcheck: en_US

\documentclass[a4paper,addpoints,12pt,answers]{exam}   

%   ============================================================================
%   CORE SETUP (Same as above)
%   ============================================================================
\usepackage[T1]{fontenc}
\usepackage{lmodern}

%% ============================================================================
%% ENHANCED TYPOGRAPHY & SPACING
%% ============================================================================
\usepackage[final]{microtype}
\microtypesetup{
    protrusion=true, 
    expansion=true
}
\UseMicrotypeSet[protrusion]{basicmath}
%   ============================================================================
%   ENHANCED EXAM-SPECIFIC GEOMETRY
%   ============================================================================
\usepackage[
    left=2cm, right=2cm, 
    top=2.5cm, bottom=2.5cm,
    headheight=25pt,
    headsep=0.8cm,
    footskip=0.8cm
]{geometry}

%% ============================================================================
%% ESSENTIAL PACKAGES (Organized by Category)
%% ============================================================================

% Math packages
\usepackage{amsmath,amssymb,amsfonts}
\usepackage{mathtools}        % Enhanced math environments
\usepackage{physics}          % Physics notation
\usepackage{siunitx}          % SI units formatting
\DeclareMathOperator\cosec{cosec}
\usepackage{tfrupee}

% Graphics and visualization
\usepackage{graphicx}
\usepackage{float}
\usepackage{caption,subcaption}
\usepackage{wrapfig}
\usepackage{multicol}

% Tables
\usepackage{booktabs}
\usepackage{tabularx,longtable}
\usepackage{tabularray}       % Modern table package

% Lists and enumeration
\usepackage{enumitem}
%\setlist[enumerate]{label=(\alph*)}  % Professional enumeration style
%\setlist[itemize]{label=\textbullet}
\usepackage{setspace}

% Colors and highlighting
\usepackage[table]{xcolor}
\definecolor{primaryblue}{RGB}{0,102,153}
\definecolor{secondarygreen}{RGB}{51,153,102}
\definecolor{accentorange}{RGB}{255,153,51}
\definecolor{warningred}{RGB}{204,51,51}
\definecolor{backgroundgray}{RGB}{248,249,250}

% TikZ and plotting
\usepackage{tikz}
\usepackage{tkz-euclide}
\usepackage{pgfplots,pgfplotstable}
\usepackage[americanvoltages,fulldiodes,siunitx]{circuitikz}

\usetikzlibrary{
    patterns, arrows.meta, calc, arrows, 
    shadows.blur, math, angles, shapes.geometric,
    positioning, decorations.pathmorphing,
    backgrounds, fit
}
\pgfplotsset{compat=1.18}
\usepgfplotslibrary{statistics,fillbetween}
\tikzset{>=Stealth}  % Modern arrow style

%% ============================================================================
%% ENHANCED TCOLORBOX ENVIRONMENTS
%% ============================================================================
\usepackage[most,breakable]{tcolorbox}
\tcbuselibrary{skins,raster,theorems}

% Professional Exam Header Box
\newtcolorbox{examheaderbox}{
    enhanced,
    colback=white,
    colframe=examblue,
    boxrule=2pt,
    arc=3mm,
    drop shadow={black!50!white},
    left=1em,
    right=1em,
    top=1em,
    bottom=1em,
    fontupper=\sffamily
}

% Instructions Box - Professional Style
\newtcolorbox{instructionsbox}{
    enhanced,
    colback=backgroundgray!30,
    colframe=headerblue,
    colbacktitle=headerblue,
    coltitle=white,
    title={\sffamily\bfseries Instructions},
    fonttitle=\sffamily\bfseries\large,
    attach boxed title to top center={yshift=-3mm},
    boxed title style={
        enhanced,
        size=small,
        colframe=headerblue,
        arc=2mm
    },
    arc=2mm,
    boxrule=1.5pt,
    drop shadow={black!20!white},
    breakable,
    left=1em,
    right=1em,
    top=1.5em,
    bottom=1em,
}

% Section Header Box
\newtcolorbox{sectionbox}[2]{
    enhanced,
    colback=primaryblue!10,
    colframe=primaryblue!80!black,
    colbacktitle=primaryblue!80!black,
    coltitle=white,
    title={\sffamily\bfseries\large SECTION #1},
    fonttitle=\sffamily\bfseries\large,
    subtitle style={
        fontupper=\sffamily\itshape,
        colback=primaryblue!5
    },
    center title,
    arc=2mm,
    boxrule=1pt,
    drop shadow={black!20!white},
    left=1em,
    right=1em,
    top=0.8em,
    bottom=0.8em,
    after title={\\\normalsize\itshape #2}
}

% Question Paper Content Box
\newtcolorbox{questionbox}{
    enhanced,
   % colback=white,
    colframe=gray!40,
    frame hidden,
    boxrule=1pt,
    arc=1mm,
    left=1em,
    right=1em,
    top=1em,
    bottom=1em,
    breakable,
      % Makes background completely transparent
    overlay={
        \draw[line width=1pt, gray!40, rounded corners=1mm] 
        (frame.north west) rectangle (frame.south east);
    }, opacityback=0, 
}

% Grade Table Box
\newtcolorbox{gradetablebox}{
    enhanced,
    colback=backgroundgray!20,
    colframe=examblue,
    boxrule=1pt,
    arc=2mm,
    left=0.5em,
    right=0.5em,
    top=0.5em,
    bottom=0.5em,
    center upper
}


%   ============================================================================
%   ENHANCED EXAM CLASS CUSTOMIZATIONS
%   ============================================================================
%  Professional question formatting
\renewcommand{\questionlabel}{\sffamily\bfseries Q.\thequestion}
\renewcommand{\partlabel}{\sffamily\bfseries(\thepartno)}
\renewcommand{\subpartlabel}{\sffamily\bfseries\thesubpart.}

%  Enhanced point display
\pointsdroppedatright
\marksnotpoints
\pointformat{\sffamily\bfseries[\themarginpoints]}
\setlength{\rightpointsmargin}{2.5cm}

%  Professional section headers for exam
\newcommand{\sectionheader}[2]{
    \begin{sectionbox}{#1}{#2}
    \end{sectionbox}
    \vspace{1em}
}

%   ============================================================================
%   ENHANCED WATERMARK FOR EXAMS
%   ============================================================================
\usepackage[firstpageonly]{draftwatermark}
\SetWatermarkText{CAJCS}
\SetWatermarkScale{0.3}
\SetWatermarkLightness{0.95}
\SetWatermarkAngle{45}

%   ============================================================================
%   SAMPLE EXAM DOCUMENT STRUCTURE
%   ============================================================================
\begin{document}

% Header and Footer Details (these exam class commands override fancyhdr)
% Keep these here as they define your specific exam header/footer.
\lhead{}
\chead{}
\rhead{}
\lfoot{}
\cfoot{Page \thepage\ of \numpages} % This correctly uses \numpages provided by exam/lastpage
\rfoot{}

% Professional Exam Header
%\begin{examheaderbox}
\begin{center}
	% \includegraphics[scale=0.4]{clogo} \\
	% \includegraphics[width=0.6\textwidth]{cajcs.JPG}   \\
	\large{\textbf{First Term Examination - September 2022}}   \\
\end{center}
\textbf{Sub: Mathematics} \hfill \textbf{Mark: 80} \hspace{1cm} \\
\textbf{Std: 10 ICSE} \hfill \textbf{Time: 3 hrs} \hspace{1cm} \\
\textbf{Date: 05-09-2022} \\
\textbf{\textit{ Reading time : 15 minutes}}

%\vspace{1em}
%	\begin{center}
%		\begin{gradetablebox}
%			\cellwidth{0.8cm}
%			\settabletotalpoints{80}
%			\hpword{Marks}
%			\gradetable[h][questions]
%		\end{gradetablebox}
%	\end{center}
%\end{examheaderbox}

\hrulefill

%\begin{instructionsbox}
	\begin{center}
		\begin{itemize}[leftmargin=0pt, itemsep=0.3em]
			\item Answers to this must be written on the paper provided separately.
			\item You will not be allowed to write during first 15 minutes.
			\item This time is to be spent in reading the question paper.
			\item The time given at the head of this paper is the time allowed for writing answers.
			\item This question paper consists of \textbf{\numpages} printed pages.
			\item[\textbf{•}] Attempt \textbf{all} questions from \textbf{Section A} and \textbf{any four} questions from \textbf{Section B}.
			\item All working, including rough work, must be clearly shown, and must be done on the same sheet as the rest of the answer.
			\item Omission of essential working will result in loss of marks.
			\item The intended marks for questions or parts of questions are given in brackets [ ].
			\item Mathematical tables are provided.
		\end{itemize}

		\vspace{1em}
%		\begin{tcolorbox}[
%				enhanced,
%				colback=warningred!10,
%				colframe=warningred!80!black,
%				arc=2mm,
%				boxrule=1pt,
%				left=0.5em,
%				right=0.5em,
%				top=0.5em,
%				bottom=0.5em
%			]
			\textbf{Instructions for Supervising Examiner:}\\
			Kindly read aloud the instructions given above to all candidates present in the examination hall.
%		\end{tcolorbox}
	\end{center}
%\end{instructionsbox}

\hrulefill

%  \sectionheader{A}{Attempt all questions}

\begin{center}
	\textbf{SECTION A} \\
	\textit{Attempt all questions}
\end{center}
%======================== Questions Begin ============================
\begin{questions}
	\doublespacing
	\pointsdroppedatright
	\marksnotpoints
	%\marginpointname{ \points}
	\setlength{\rightpointsmargin}{2.1cm}
	\pointformat{\bfseries\boldmath[\themarginpoints]}
	% Question 1
	\question[15] Choose the correct answers to the questions from the given options. \droppoints
	\begin{parts}
		\renewcommand{\thepartno}{\roman{partno}}
		%MCQ 1
		\part this is new file.

		\begin{oneparchoices}
			\choice
			\choice
			\choice
			\choice
		\end{oneparchoices}
		%MCQ 2
		\part

		\begin{oneparchoices}
			\choice
			\choice
			\choice
			\choice
		\end{oneparchoices}
		%MCQ 3
		\part

		\begin{oneparchoices}
			\choice
			\choice
			\choice
			\choice
		\end{oneparchoices}
		%MCQ 4
		\part

		\begin{oneparchoices}
			\choice
			\choice
			\choice
			\choice
		\end{oneparchoices}
		%MCQ 5
		\part

		\begin{oneparchoices}
			\choice
			\choice
			\choice
			\choice
		\end{oneparchoices}
		%MCQ 6
		\part

		\begin{oneparchoices}
			\choice
			\choice
			\choice
			\choice
		\end{oneparchoices}
		%MCQ 7
		\part

		\begin{oneparchoices}
			\choice
			\choice
			\choice
			\choice
		\end{oneparchoices}
		%MCQ 8
		\part

		\begin{oneparchoices}
			\choice
			\choice
			\choice
			\choice
		\end{oneparchoices}
		%MCQ 9
		\part

		\begin{oneparchoices}
			\choice
			\choice
			\choice
			\choice
		\end{oneparchoices}
		%MCQ 10
		\part

		\begin{oneparchoices}
			\choice
			\choice
			\choice
			\choice
		\end{oneparchoices}
		% 11
		\part

		\begin{oneparchoices}
			\choice
			\choice
			\choice
			\choice
		\end{oneparchoices}
		% 12
		\part

		\begin{oneparchoices}
			\choice
			\choice
			\choice
			\choice
		\end{oneparchoices}
		% 13
		\part

		\begin{oneparchoices}
			\choice
			\choice
			\choice
			\choice
		\end{oneparchoices}
		% 14
		\part

		\begin{oneparchoices}
			\choice
			\choice
			\choice
			\choice
		\end{oneparchoices}
		% 15
		\part

		\begin{oneparchoices}
			\choice
			\choice
			\choice
			\choice
		\end{oneparchoices}
	\end{parts}

	% Question 2
	\question
	\begin{parts}
		\part[4]    \droppoints
		\part[4]    \droppoints
		\part[4]    \droppoints
	\end{parts}
	% Question 3
	\question
	\begin{parts}
		\part[4]    \droppoints
		\part[4]    \droppoints
		\part[4]    \droppoints
	\end{parts}

% Section B Header
%		\sectionheader{B}{Attempt any 4 questions (40 marks)}
		
	\begin{center}
		\textbf{SECTION B (40 marks)} \\
		\textbf{Attempt any 4 questions }
	\end{center}
	% Question 4
	\question
	\begin{parts}
		\part[3]    \droppoints
		\part[3]    \droppoints
		\part[4]    \droppoints
	\end{parts}

	% Question 5
	\question
	\begin{parts}
		\part[3]    \droppoints
		\part[3]    \droppoints
		\part[4]    \droppoints
	\end{parts}

	% Question 6
	\question
	\begin{parts}
		\part[3]    \droppoints
		\part[3]    \droppoints
		\part[4]    \droppoints
	\end{parts}

	% Question 7
	\question
	\begin{parts}
		\part[3]    \droppoints
		\part[3]    \droppoints
		\part[4]    \droppoints
	\end{parts}

	% Question 8
	\question
	\begin{parts}
		\part[3]    \droppoints
		\part[3]    \droppoints
		\part[4]    \droppoints
	\end{parts}

	% Question 9
	\question
	\begin{parts}
		\part[3]    \droppoints
		\part[3]    \droppoints
		\part[4]    \droppoints
	\end{parts}
	% Question 10
	\question
	\begin{parts}
		\part[4]    \droppoints
		\part[6]    \droppoints
	\end{parts}

\end{questions}
\centering
\vspace{2cm}
********
\end{document}
